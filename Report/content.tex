\section{Типы шумов}

\textbf{Шум} -- разнообразные искажения на цифровых изображениях, обусловленные разного рода помехами.

В данной лабораторной работе мы рассмотрим наиболее распространенные модели шумов на примере воздействия их на изображения \ref{img:source}. 

\begin{figure}[ht!!]
    \centering
    \includegraphics[width=\textwidth]{../lewis-hine-taschen-main-3.jpg}
    \caption{Исходное изображение}
    \label{img:source}  
\end{figure}
\FloatBarrier

\subsection{Импульсный шум}
Зашумленное изображение $I$ описывается следующей системой, причем значение интенсивности пикселя $I(x,y)$ будет изменено на значение $d \in [0,255]$:
\begin{equation}
    \begin{cases} d,\, \text{c вероятностью}\, p, \\
    s_{x,y},\, \text{c вероятностью}\, (1-p),
    \end{cases}
\end{equation}

где $s_{x,y}$ — интенсивность пикселя исходного изображения, если $d = 0$ — шум типа «перец», если
$d = 255$ — шум типа «соль».

\begin{figure}[ht!!]
    \centering
    \includegraphics[width=\textwidth]{../Noisy_images/Impulse_noise.jpg}
    \caption{Результат воздействия импульсного шума на изображение \ref{img:source}}
    \label{img:impulse_noise}  
\end{figure}
\FloatBarrier

Итак, на изображении \ref{img:impulse_noise} отчетливо видны появившиеся белые («соль») и черные («перец») точки, что является характерным для импульсного шума, именно из-за этого он часто называется точечным шумом

\subsection{Аддитивный шум}
Аддитивный шум описывается следующим выражением:
\begin{equation}
    I_{new}(x,y) = I(x,y)  + \eta(x,y),
\end{equation}
где $ I_{new}$ — зашумленное изображение, $I$ — исходное изображение, $\eta$ — не зависящий от сигнала аддитивный шум с гауссовым или любым другим распределением функции плотности вероятности.

\begin{figure}[ht!!]
    \centering
    \includegraphics[width=\textwidth]{../Noisy_images/Additive_noise.jpg}
    \caption{Результат воздействия аддитивного шума на изображение \ref{img:source}}
    \label{img:additive_noise}
\end{figure}
\FloatBarrier

Наложив аддитивный шум на исходное изображение, мы получили как будто выцветшую картинку, на которой также появилась небольшая зернистость

\subsection{Мультипликативный шум}
Мультипликативный шум описывается следующим выражением:
\begin{equation}
    I_{new}(x,y) = I(x,y) \cdot \eta(x,y),
\end{equation}

Частным случаем мультипликативного шума является спекл-шум, который мы и рассмотрим

\begin{figure}[ht!!]
    \centering
    \includegraphics[width=\textwidth]{../Noisy_images/Speckle_noise.jpg}
    \caption{Результат воздействия спекл-шума на изображение \ref{img:source}}
    \label{img:speckle_noise}
\end{figure}
\FloatBarrier

Мы получили изображение на котором можно отчетливо наблюдать светлые пятна, крапинки (спеклы), которые разделены темными участками изображения, что соответственно характерно при наложении спекл-шума

\subsection{Гауссов (нормальный) шум}
Функция распределения плотности вероятности
$p(z)$ случайной величины $z$ описывается следующим выражением:
\begin{equation}
    p(z)= \frac{1}{\sigma\sqrt{2\pi}}\, e^{\frac{-(z-\mu)^2}{2\sigma^2}},
\end{equation}
где $z$ — интенсивность изображения (например, для полутонового изображения $z \in [0,255]$), $\eta$ — среднее (математическое ожидание) случайной величины $z$, $\sigma$ — среднеквадратичное отклонение, дисперсия $\sigma^2$ определяет мощность вносимого шума.


\begin{figure}[ht!!]
    \centering
    \includegraphics[width=\textwidth]{../Noisy_images/Gaussian_noise.jpg}
    \caption{Результат воздействия Гауссова шума на изображение \ref{img:source}}
    \label{img:gaussian_noise}
\end{figure}
\FloatBarrier

При применении Гауссового (нормального) шума к изображению, оно становится более размытым или зернистым. Он добавляет случайные значения пикселям изображения, что приводит к потере деталей и четкости.

\subsection{Шум квантования}
Приближенно шум квантования можно описать распределением Пуассона. Такой шум не устраняется.

\begin{figure}[ht!!]
    \centering
    \includegraphics[width=\textwidth]{../Noisy_images/Poisson_Noise.jpg}
    \caption{Результат воздействия шума квантования на изображение \ref{img:source}}
    \label{img:poisson_noise}
\end{figure}
\FloatBarrier

Шум на первый взгляд может показаться незаметным, но он есть, при сильном растяжении изображения это видно. На картинке появилась очень мелкая зернистость

\section{Фильтрация изображений}
\textit{Локальным} преобразованием называется такое преобразование, при котором для вычисления значения интенсивности каждого пикселя учитываются значения соседних пикселей в некоторой окрестности, называемой \textit{окном}, представляющей собой некоторую матрицу, которую также называют \textit{маской, фильтром, ядром фильтра}, а сами значения элементов
матрицы соответсвенно \textit{коэффициентами}. Как правило, маска имеет квадратную форму.

Фильтрация изображения $I$, имеющего размеры $M \times N$, с помощью маски размера $m \times n$ описывается формулой:
\begin{equation}
    I_{new}(x,y) = \sum_s \sum_t w(s,t) I(x+s,y+t),
\end{equation}
где $s$ и $t$ — координаты элементов маски относительно ее центра (в
центре $s = t = 0$). Такого рода преобразования называются \textit{линейными}.

\textit{Фильтрация в скользящем окне} — преобразование, при котором после вычисления нового значения интенсивности пикселя $I_{new}(x,y)$ окно $w$, в котором описана маска фильтра, сдвигается и
вычисляется интенсивность следующего пикселя.

\subsection{Низкочастотная фильтрация}
Низкочастотные пространственные фильтры ослабляют высокочастотные компоненты (области с сильным изменением интенсивностей) и оставляют низкочастотные компоненты изображения
без изменений. Отличительными особенностями
ядра низкочастотного фильтра являются: неотрицательные коэффициенты маски и то, что сумма всех коэффициентов равна единице.

\subsubsection{Контргармонический усредняющий фильтр}
Фильтр базируется на выражении:
\begin{equation}
    I_{new}(x,y) = \frac{\sum_{i=0}^m \sum_{j=0}^n I(i,j)^{Q+1}}{\sum_{i=0}^m \sum_{j=0}^n I(i,j)^Q}, \: \text{где $Q$ — порядок фильтра.}
\end{equation}


Рассмотрим применение фильтра при $Q > 0$ и $Q < 0$ к различным типам шумов. 

\begin{figure}[ht!] 
    \centering
    \begin{subfigure}[b]{0.5\linewidth}
        \centering
        \includegraphics[width=0.95\linewidth]{../lewis-hine-taschen-main-3.jpg} 
        \caption{Исходное изображение} 
        \label{contraharmonic_-1.85:a} 
        \vspace{4ex}
    \end{subfigure}%%
    \begin{subfigure}[b]{0.5\linewidth}
      \centering
      \includegraphics[width=0.95\linewidth]{../Contraharmonic_Filter/Contraharmonic_Impulse_noise_(m,n=(3,_3),q=-1.85).jpg} 
      \caption{Импульсный шум} 
      \label{contraharmonic_-1.85:b} 
      \vspace{4ex}
    \end{subfigure}
    \begin{subfigure}[b]{0.5\linewidth}
      \centering
      \includegraphics[width=0.95\linewidth]{../Contraharmonic_Filter/Contraharmonic_Additive_noise_(m,n=(3,_3),q=-1.85).jpg} 
      \caption{Аддитивный шум} 
      \label{contraharmonic_-1.85:c} 
      \vspace{4ex}
    \end{subfigure}%%
    \begin{subfigure}[b]{0.5\linewidth}
      \centering
      \includegraphics[width=0.95\linewidth]{../Contraharmonic_Filter/Contraharmonic_Gaussian_noise_(m,n=(3,_3),q=-1.85).jpg} 
      \caption{Гауссов шум} 
      \label{contraharmonic_-1.85:d} 
      \vspace{4ex}
    \end{subfigure}
    \begin{subfigure}[b]{0.5\linewidth}
      \centering
      \includegraphics[width=0.95\linewidth]{../Contraharmonic_Filter/Contraharmonic_Poisson_noise_(m,n=(3,_3),q=-1.85).jpg} 
      \caption{Шум квантования} 
      \label{contraharmonic_-1.85:e}
    \end{subfigure}%% 
    \begin{subfigure}[b]{0.5\linewidth}
        \centering
        \includegraphics[width=0.95\linewidth]{../Contraharmonic_Filter/Contraharmonic_Speckle_noise_(m,n=(3,_3),q=-1.85).jpg} 
        \caption{Мультипликативный шум} 
        \label{contraharmonic_-1.85:f} 
    \end{subfigure} 
    \caption{Результат применения контргармонического усредняющего фильтра при значении $Q = -1.85$ к различным типам шумов}
    \label{img:contraharmonic_-1.85} 
  \end{figure}

  \begin{figure}[ht!] 
    \centering
    \begin{subfigure}[b]{0.5\linewidth}
        \centering
        \includegraphics[width=0.95\linewidth]{../lewis-hine-taschen-main-3.jpg} 
        \caption{Исходное изображение} 
        \label{contraharmonic_-0.85:a} 
        \vspace{4ex}
    \end{subfigure}%%
    \begin{subfigure}[b]{0.5\linewidth}
      \centering
      \includegraphics[width=0.95\linewidth]{../Contraharmonic_Filter/Contraharmonic_Impulse_noise_(m,n=(3,_3),q=-0.85).jpg} 
      \caption{Импульсный шум} 
      \label{contraharmonic_-0.85:b} 
      \vspace{4ex}
    \end{subfigure}
    \begin{subfigure}[b]{0.5\linewidth}
      \centering
      \includegraphics[width=0.95\linewidth]{../Contraharmonic_Filter/Contraharmonic_Additive_noise_(m,n=(3,_3),q=-0.85).jpg} 
      \caption{Аддитивный шум} 
      \label{contraharmonic_-0.85:c} 
      \vspace{4ex}
    \end{subfigure}%%
    \begin{subfigure}[b]{0.5\linewidth}
      \centering
      \includegraphics[width=0.95\linewidth]{../Contraharmonic_Filter/Contraharmonic_Gaussian_noise_(m,n=(3,_3),q=-0.85).jpg} 
      \caption{Гауссов шум} 
      \label{contraharmonic_-0.85:d} 
      \vspace{4ex}
    \end{subfigure}
    \begin{subfigure}[b]{0.5\linewidth}
      \centering
      \includegraphics[width=0.95\linewidth]{../Contraharmonic_Filter/Contraharmonic_Poisson_noise_(m,n=(3,_3),q=-0.85).jpg} 
      \caption{Шум квантования} 
      \label{contraharmonic_-0.85:e}
    \end{subfigure}%% 
    \begin{subfigure}[b]{0.5\linewidth}
        \centering
        \includegraphics[width=0.95\linewidth]{../Contraharmonic_Filter/Contraharmonic_Speckle_noise_(m,n=(3,_3),q=-0.85).jpg} 
        \caption{Мультипликативный шум} 
        \label{contraharmonic_-0.85:f} 
    \end{subfigure} 
    \caption{Результат применения контргармонического усредняющего фильтра при значении $Q = -0.85$ к различным типам шумов}
    \label{img:contraharmonic_-0.85} 
  \end{figure}

  \begin{figure}[ht!] 
    \centering
    \begin{subfigure}[b]{0.5\linewidth}
        \centering
        \includegraphics[width=0.95\linewidth]{../lewis-hine-taschen-main-3.jpg} 
        \caption{Исходное изображение} 
        \label{contraharmonic_-0.5:a} 
        \vspace{4ex}
    \end{subfigure}%%
    \begin{subfigure}[b]{0.5\linewidth}
      \centering
      \includegraphics[width=0.95\linewidth]{../Contraharmonic_Filter/Contraharmonic_Impulse_noise_(m,n=(3,_3),q=-0.5).jpg} 
      \caption{Импульсный шум} 
      \label{contraharmonic_-0.5:b} 
      \vspace{4ex}
    \end{subfigure}
    \begin{subfigure}[b]{0.5\linewidth}
      \centering
      \includegraphics[width=0.95\linewidth]{../Contraharmonic_Filter/Contraharmonic_Additive_noise_(m,n=(3,_3),q=-0.5).jpg} 
      \caption{Аддитивный шум} 
      \label{contraharmonic_-0.5:c} 
      \vspace{4ex}
    \end{subfigure}%%
    \begin{subfigure}[b]{0.5\linewidth}
      \centering
      \includegraphics[width=0.95\linewidth]{../Contraharmonic_Filter/Contraharmonic_Gaussian_noise_(m,n=(3,_3),q=-0.5).jpg} 
      \caption{Гауссов шум} 
      \label{contraharmonic_-0.5:d} 
      \vspace{4ex}
    \end{subfigure}
    \begin{subfigure}[b]{0.5\linewidth}
      \centering
      \includegraphics[width=0.95\linewidth]{../Contraharmonic_Filter/Contraharmonic_Poisson_noise_(m,n=(3,_3),q=-0.5).jpg} 
      \caption{Шум квантования} 
      \label{contraharmonic_-0.5:e}
    \end{subfigure}%% 
    \begin{subfigure}[b]{0.5\linewidth}
        \centering
        \includegraphics[width=0.95\linewidth]{../Contraharmonic_Filter/Contraharmonic_Speckle_noise_(m,n=(3,_3),q=-0.5).jpg} 
        \caption{Мультипликативный шум} 
        \label{contraharmonic_-0.5:f} 
    \end{subfigure} 
    \caption{Результат применения контргармонического усредняющего фильтра при значении $Q = -0.5$ к различным типам шумов}
    \label{img:contraharmonic_-0.5} 
  \end{figure}

  \begin{figure}[ht!] 
    \centering
    \begin{subfigure}[b]{0.5\linewidth}
        \centering
        \includegraphics[width=0.95\linewidth]{../lewis-hine-taschen-main-3.jpg} 
        \caption{Исходное изображение} 
        \label{contraharmonic_0.5:a} 
        \vspace{4ex}
    \end{subfigure}%%
    \begin{subfigure}[b]{0.5\linewidth}
      \centering
      \includegraphics[width=0.95\linewidth]{../Contraharmonic_Filter/Contraharmonic_Impulse_noise_(m,n=(3,_3),q=0.5).jpg} 
      \caption{Импульсный шум} 
      \label{contraharmonic_0.5:b} 
      \vspace{4ex}
    \end{subfigure}
    \begin{subfigure}[b]{0.5\linewidth}
      \centering
      \includegraphics[width=0.95\linewidth]{../Contraharmonic_Filter/Contraharmonic_Additive_noise_(m,n=(3,_3),q=0.5).jpg} 
      \caption{Аддитивный шум} 
      \label{contraharmonic_0.5:c} 
      \vspace{4ex}
    \end{subfigure}%%
    \begin{subfigure}[b]{0.5\linewidth}
      \centering
      \includegraphics[width=0.95\linewidth]{../Contraharmonic_Filter/Contraharmonic_Gaussian_noise_(m,n=(3,_3),q=0.5).jpg} 
      \caption{Гауссов шум} 
      \label{contraharmonic_0.5:d} 
      \vspace{4ex}
    \end{subfigure}
    \begin{subfigure}[b]{0.5\linewidth}
      \centering
      \includegraphics[width=0.95\linewidth]{../Contraharmonic_Filter/Contraharmonic_Poisson_noise_(m,n=(3,_3),q=0.5).jpg} 
      \caption{Шум квантования} 
      \label{contraharmonic_0.5:e}
    \end{subfigure}%% 
    \begin{subfigure}[b]{0.5\linewidth}
        \centering
        \includegraphics[width=0.95\linewidth]{../Contraharmonic_Filter/Contraharmonic_Speckle_noise_(m,n=(3,_3),q=0.5).jpg} 

        \caption{Мультипликативный шум} 
        \label{contraharmonic_0.5:f} 
    \end{subfigure} 
    \caption{Результат применения контргармонического усредняющего фильтра при значении $Q = 0.5$ к различным типам шумов}
    \label{img:contraharmonic_0.5} 
  \end{figure}

  \begin{figure}[ht!] 
    \centering
    \begin{subfigure}[b]{0.5\linewidth}
        \centering
        \includegraphics[width=0.95\linewidth]{../lewis-hine-taschen-main-3.jpg} 
        \caption{Исходное изображение} 
        \label{contraharmonic_0.85:a} 
        \vspace{4ex}
    \end{subfigure}%%
    \begin{subfigure}[b]{0.5\linewidth}
      \centering
      \includegraphics[width=0.95\linewidth]{../Contraharmonic_Filter/Contraharmonic_Impulse_noise_(m,n=(3,_3),q=0.85).jpg} 
      \caption{Импульсный шум} 
      \label{contraharmonic_0.85:b} 
      \vspace{4ex}
    \end{subfigure}
    \begin{subfigure}[b]{0.5\linewidth}
      \centering
      \includegraphics[width=0.95\linewidth]{../Contraharmonic_Filter/Contraharmonic_Additive_noise_(m,n=(3,_3),q=0.85).jpg} 
      \caption{Аддитивный шум} 
      \label{contraharmonic_0.85:c} 
      \vspace{4ex}
    \end{subfigure}%%
    \begin{subfigure}[b]{0.5\linewidth}
      \centering
      \includegraphics[width=0.95\linewidth]{../Contraharmonic_Filter/Contraharmonic_Gaussian_noise_(m,n=(3,_3),q=0.85).jpg} 
      \caption{Гауссов шум} 
      \label{contraharmonic_0.85:d} 
      \vspace{4ex}
    \end{subfigure}
    \begin{subfigure}[b]{0.5\linewidth}
      \centering
      \includegraphics[width=0.95\linewidth]{../Contraharmonic_Filter/Contraharmonic_Poisson_noise_(m,n=(3,_3),q=0.85).jpg} 
      \caption{Шум квантования} 
      \label{contraharmonic_0.85:e}
    \end{subfigure}%% 
    \begin{subfigure}[b]{0.5\linewidth}
        \centering
        \includegraphics[width=0.95\linewidth]{../Contraharmonic_Filter/Contraharmonic_Speckle_noise_(m,n=(3,_3),q=0.85).jpg} 
        \caption{Мультипликативный шум} 
        \label{contraharmonic_0.85:f} 
    \end{subfigure} 
    \caption{Результат применения контргармонического усредняющего фильтра при значении $Q = 0.85$ к различным типам шумов}
    \label{img:contraharmonic_0.85} 
  \end{figure}

  \begin{figure}[ht!] 
    \centering
    \begin{subfigure}[b]{0.5\linewidth}
        \centering
        \includegraphics[width=0.95\linewidth]{../lewis-hine-taschen-main-3.jpg} 
        \caption{Исходное изображение} 
        \label{contraharmonic_1.85:a} 
        \vspace{4ex}
    \end{subfigure}%%
    \begin{subfigure}[b]{0.5\linewidth}
      \centering
      \includegraphics[width=0.95\linewidth]{../Contraharmonic_Filter/Contraharmonic_Impulse_noise_(m,n=(3,_3),q=1.85).jpg} 
      \caption{Импульсный шум} 
      \label{contraharmonic_1.85:b} 
      \vspace{4ex}
    \end{subfigure}
    \begin{subfigure}[b]{0.5\linewidth}
      \centering
      \includegraphics[width=0.95\linewidth]{../Contraharmonic_Filter/Contraharmonic_Additive_noise_(m,n=(3,_3),q=1.85).jpg} 
      \caption{Аддитивный шум} 
      \label{contraharmonic_1.85:c} 
      \vspace{4ex}
    \end{subfigure}%%
    \begin{subfigure}[b]{0.5\linewidth}
      \centering
      \includegraphics[width=0.95\linewidth]{../Contraharmonic_Filter/Contraharmonic_Gaussian_noise_(m,n=(3,_3),q=1.85).jpg} 
      \caption{Гауссов шум} 
      \label{contraharmonic_1.85:d} 
      \vspace{4ex}
    \end{subfigure}
    \begin{subfigure}[b]{0.5\linewidth}
      \centering
      \includegraphics[width=0.95\linewidth]{../Contraharmonic_Filter/Contraharmonic_Poisson_noise_(m,n=(3,_3),q=1.85).jpg} 
      \caption{Шум квантования} 
      \label{contraharmonic_1.85:e}
    \end{subfigure}%% 
    \begin{subfigure}[b]{0.5\linewidth}
        \centering
        \includegraphics[width=0.95\linewidth]{../Contraharmonic_Filter/Contraharmonic_Speckle_noise_(m,n=(3,_3),q=1.85).jpg} 
        \caption{Мультипликативный шум} 
        \label{contraharmonic_1.85:f} 
    \end{subfigure} 
    \caption{Результат применения контргармонического усредняющего фильтра при значении $Q = 1.85$ к различным типам шумов}
    \label{img:contraharmonic_1.85} 
  \end{figure}

  \FloatBarrier

При контргармоническом фильтре одновременное удаление белых и черных точек невозможно, такой фильтр является обобщением усредняющих фильтров. 
Результаты обработки доказывают, что при $Q$ < 0 подавляются шумы типа «соль», и заметно, что изображения становятся темнее, а при $Q$ > 0 подавляются шумы типа «перец», и картинки будто обесцвечивают, причем чем больше значения мы берем, тем больше виден эффект.

\subsubsection{Фильтр Гаусса}
При фильтрации изображений будем использовать двумерный фильтр Гаусса:
\begin{equation}
   G_\sigma = \frac{1}{2\pi\sigma^2} e^{-\frac{x^2+y^2}{2\sigma^2}} = \frac{1}{\sigma\sqrt{2\pi}} e^{\frac{-x^2}{2\sigma^2}} \cdot \frac{1}{\sigma\sqrt{2\pi}} e^{\frac{-y^2}{2\sigma^2}}
\end{equation}

\begin{figure}[ht!] 
    \centering
    \begin{subfigure}[b]{0.5\linewidth}
        \centering
        \includegraphics[width=0.95\linewidth]{../lewis-hine-taschen-main-3.jpg} 
        \caption{Исходное изображение} 
        \label{gaussian_3:a} 
        \vspace{4ex}
    \end{subfigure}%%
    \begin{subfigure}[b]{0.5\linewidth}
      \centering
      \includegraphics[width=0.95\linewidth]{../Gaussian_Blur/Gaussian_Blur_Impulse_noise_(3,3).jpg} 
      \caption{Импульсный шум} 
      \label{gaussian_3:b} 
      \vspace{4ex}
    \end{subfigure}
    \begin{subfigure}[b]{0.5\linewidth}
      \centering
      \includegraphics[width=0.95\linewidth]{../Gaussian_Blur/Gaussian_Blur_Additive_noise_(3,3).jpg} 
      \caption{Аддитивный шум} 
      \label{gaussian_3:c} 
      \vspace{4ex}
    \end{subfigure}%%
    \begin{subfigure}[b]{0.5\linewidth}
      \centering
      \includegraphics[width=0.95\linewidth]{../Gaussian_Blur/Gaussian_Blur_Gaussian_noise_(3,3).jpg} 
      \caption{Гауссов шум} 
      \label{gaussian_3:d} 
      \vspace{4ex}
    \end{subfigure}
    \begin{subfigure}[b]{0.5\linewidth}
      \centering
      \includegraphics[width=0.95\linewidth]{../Gaussian_Blur/Gaussian_Blur_Poisson_noise_(3,3).jpg} 
      \caption{Шум квантования} 
      \label{gaussian_3:e}
    \end{subfigure}%% 
    \begin{subfigure}[b]{0.5\linewidth}
        \centering
        \includegraphics[width=0.95\linewidth]{../Gaussian_Blur/Gaussian_Blur_Speckle_noise_(3,3).jpg} 
        \caption{Мультипликативный шум} 
        \label{gaussian_3:f} 
    \end{subfigure} 
    \caption{Результат применения фильтра Гаусса к различным типам шумов при $n = 3$}
    \label{img:gaussian_3} 
  \end{figure}

  \begin{figure}[ht!] 
    \centering
    \begin{subfigure}[b]{0.5\linewidth}
        \centering
        \includegraphics[width=0.95\linewidth]{../lewis-hine-taschen-main-3.jpg} 
        \caption{Исходное изображение} 
        \label{gaussian_7:a} 
        \vspace{4ex}
    \end{subfigure}%%
    \begin{subfigure}[b]{0.5\linewidth}
      \centering
      \includegraphics[width=0.95\linewidth]{../Gaussian_Blur/Gaussian_Blur_Impulse_noise_(7,7).jpg} 
      \caption{Импульсный шум} 
      \label{gaussian_7:b} 
      \vspace{4ex}
    \end{subfigure}
    \begin{subfigure}[b]{0.5\linewidth}
      \centering
      \includegraphics[width=0.95\linewidth]{../Gaussian_Blur/Gaussian_Blur_Additive_noise_(7,7).jpg} 
      \caption{Аддитивный шум} 
      \label{gaussian_7:c} 
      \vspace{4ex}
    \end{subfigure}%%
    \begin{subfigure}[b]{0.5\linewidth}
      \centering
      \includegraphics[width=0.95\linewidth]{../Gaussian_Blur/Gaussian_Blur_Gaussian_noise_(7,7).jpg} 
      \caption{Гауссов шум} 
      \label{gaussian_7:d} 
      \vspace{4ex}
    \end{subfigure}
    \begin{subfigure}[b]{0.5\linewidth}
      \centering
      \includegraphics[width=0.95\linewidth]{../Gaussian_Blur/Gaussian_Blur_Poisson_noise_(7,7).jpg} 
      \caption{Шум квантования} 
      \label{gaussian_7:e}
    \end{subfigure}%% 
    \begin{subfigure}[b]{0.5\linewidth}
        \centering
        \includegraphics[width=0.95\linewidth]{../Gaussian_Blur/Gaussian_Blur_Speckle_noise_(7,7).jpg} 
        \caption{Мультипликативный шум} 
        \label{gaussian_7:f} 
    \end{subfigure} 
    \caption{Результат применения фильтра Гаусса к различным типам шумов при $n = 7$}
    \label{img:gaussian_7} 
  \end{figure}

  \begin{figure}[ht!] 
    \centering
    \begin{subfigure}[b]{0.5\linewidth}
        \centering
        \includegraphics[width=0.95\linewidth]{../lewis-hine-taschen-main-3.jpg} 
        \caption{Исходное изображение} 
        \label{gaussian_11:a} 
        \vspace{4ex}
    \end{subfigure}%%
    \begin{subfigure}[b]{0.5\linewidth}
      \centering
      \includegraphics[width=0.95\linewidth]{../Gaussian_Blur/Gaussian_Blur_Impulse_noise_(11,11).jpg} 
      \caption{Импульсный шум} 
      \label{gaussian_11:b} 
      \vspace{4ex}
    \end{subfigure}
    \begin{subfigure}[b]{0.5\linewidth}
      \centering
      \includegraphics[width=0.95\linewidth]{../Gaussian_Blur/Gaussian_Blur_Additive_noise_(11,11).jpg} 
      \caption{Аддитивный шум} 
      \label{gaussian_11:c} 
      \vspace{4ex}
    \end{subfigure}%%
    \begin{subfigure}[b]{0.5\linewidth}
      \centering
      \includegraphics[width=0.95\linewidth]{../Gaussian_Blur/Gaussian_Blur_Gaussian_noise_(11,11).jpg} 
      \caption{Гауссов шум} 
      \label{gaussian_11:d} 
      \vspace{4ex}
    \end{subfigure}
    \begin{subfigure}[b]{0.5\linewidth}
      \centering
      \includegraphics[width=0.95\linewidth]{../Gaussian_Blur/Gaussian_Blur_Poisson_noise_(11,11).jpg} 
      \caption{Шум квантования} 
      \label{gaussian_11:e}
    \end{subfigure}%% 
    \begin{subfigure}[b]{0.5\linewidth}
        \centering
        \includegraphics[width=0.95\linewidth]{../Gaussian_Blur/Gaussian_Blur_Speckle_noise_(11,11).jpg} 
        \caption{Мультипликативный шум} 
        \label{gaussian_11:f} 
    \end{subfigure} 
    \caption{Результат применения фильтра Гаусса к различным типам шумов при $n = 11$}
    \label{img:gaussian_11} 
\end{figure}
  \FloatBarrier

Рассмотрев применение фильтра Гаусса при различных $n$, можно сделать вывод: чем больше значение $n$ мы берём больше получаем эффект размытия на изображениях.

\subsection{Нелинейная фильтрация}

\subsubsection{Медианная фильтрация}

\begin{figure}[ht!] 
    \centering
    \begin{subfigure}[b]{0.5\linewidth}
        \centering
        \includegraphics[width=0.95\linewidth]{../lewis-hine-taschen-main-3.jpg} 
        \caption{Исходное изображение} 
        \label{median_3:a} 
        \vspace{4ex}
    \end{subfigure}%%
    \begin{subfigure}[b]{0.5\linewidth}
      \centering
      \includegraphics[width=0.95\linewidth]{../Median_FIlter/Median_Impulse_noise_(k=3).jpg} 
      \caption{Импульсный шум} 
      \label{median_3:b} 
      \vspace{4ex}
    \end{subfigure}
    \begin{subfigure}[b]{0.5\linewidth}
      \centering
      \includegraphics[width=0.95\linewidth]{../Median_FIlter/Median_Additive_noise_(k=3).jpg} 
      \caption{Аддитивный шум} 
      \label{median_3:c} 
      \vspace{4ex}
    \end{subfigure}%%
    \begin{subfigure}[b]{0.5\linewidth}
      \centering
      \includegraphics[width=0.95\linewidth]{../Median_FIlter/Median_Gaussian_noise_(k=3).jpg} 
      \caption{Гауссов шум} 
      \label{median_3:d} 
      \vspace{4ex}
    \end{subfigure}
    \begin{subfigure}[b]{0.5\linewidth}
      \centering
      \includegraphics[width=0.95\linewidth]{../Median_FIlter/Median_Poisson_noise_(k=3).jpg} 
      \caption{Шум квантования} 
      \label{median_3:e}
    \end{subfigure}%% 
    \begin{subfigure}[b]{0.5\linewidth}
        \centering
        \includegraphics[width=0.95\linewidth]{../Median_FIlter/Median_Speckle_noise_(k=3).jpg} 
        \caption{Мультипликативный шум} 
        \label{median_3:f} 
    \end{subfigure} 
    \caption{Результат применения медианной фильтрации к различным типам шумов при $k = 3$}
    \label{img:median_3} 
\end{figure}

\begin{figure}[ht!] 
    \centering
    \begin{subfigure}[b]{0.5\linewidth}
        \centering
        \includegraphics[width=0.95\linewidth]{../lewis-hine-taschen-main-3.jpg} 
        \caption{Исходное изображение} 
        \label{median_5:a} 
        \vspace{4ex}
    \end{subfigure}%%
    \begin{subfigure}[b]{0.5\linewidth}
      \centering
      \includegraphics[width=0.95\linewidth]{../Median_FIlter/Median_Impulse_noise_(k=5).jpg} 
      \caption{Импульсный шум} 
      \label{median_5:b} 
      \vspace{4ex}
    \end{subfigure}
    \begin{subfigure}[b]{0.5\linewidth}
      \centering
      \includegraphics[width=0.95\linewidth]{../Median_FIlter/Median_Additive_noise_(k=5).jpg} 
      \caption{Аддитивный шум} 
      \label{median_5:c} 
      \vspace{4ex}
    \end{subfigure}%%
    \begin{subfigure}[b]{0.5\linewidth}
      \centering
      \includegraphics[width=0.95\linewidth]{../Median_FIlter/Median_Gaussian_noise_(k=5).jpg} 
      \caption{Гауссов шум} 
      \label{median_5:d} 
      \vspace{4ex}
    \end{subfigure}
    \begin{subfigure}[b]{0.5\linewidth}
      \centering
      \includegraphics[width=0.95\linewidth]{../Median_FIlter/Median_Poisson_noise_(k=5).jpg} 
      \caption{Шум квантования} 
      \label{median_5:e}
    \end{subfigure}%% 
    \begin{subfigure}[b]{0.5\linewidth}
        \centering
        \includegraphics[width=0.95\linewidth]{../Median_FIlter/Median_Speckle_noise_(k=5).jpg} 
        \caption{Мультипликативный шум} 
        \label{median_5:f} 
    \end{subfigure} 
    \caption{Результат применения медианной фильтрации к различным типам шумов при $k = 5$}
    \label{img:median_5} 
\end{figure}

\begin{figure}[ht!] 
    \centering
    \begin{subfigure}[b]{0.5\linewidth}
        \centering
        \includegraphics[width=0.95\linewidth]{../lewis-hine-taschen-main-3.jpg} 
        \caption{Исходное изображение} 
        \label{median_7:a} 
        \vspace{4ex}
    \end{subfigure}%%
    \begin{subfigure}[b]{0.5\linewidth}
      \centering
      \includegraphics[width=0.95\linewidth]{../Median_FIlter/Median_Impulse_noise_(k=7).jpg} 
      \caption{Импульсный шум} 
      \label{median_7:b} 
      \vspace{4ex}
    \end{subfigure}
    \begin{subfigure}[b]{0.5\linewidth}
      \centering
      \includegraphics[width=0.95\linewidth]{../Median_FIlter/Median_Additive_noise_(k=7).jpg} 
      \caption{Аддитивный шум} 
      \label{median_7:c} 
      \vspace{4ex}
    \end{subfigure}%%
    \begin{subfigure}[b]{0.5\linewidth}
      \centering
      \includegraphics[width=0.95\linewidth]{../Median_FIlter/Median_Gaussian_noise_(k=7).jpg} 
      \caption{Гауссов шум} 
      \label{median_7:d} 
      \vspace{4ex}
    \end{subfigure}
    \begin{subfigure}[b]{0.5\linewidth}
      \centering
      \includegraphics[width=0.95\linewidth]{../Median_FIlter/Median_Poisson_noise_(k=7).jpg} 
      \caption{Шум квантования} 
      \label{median_7:e}
    \end{subfigure}%% 
    \begin{subfigure}[b]{0.5\linewidth}
        \centering
        \includegraphics[width=0.95\linewidth]{../Median_FIlter/Median_Speckle_noise_(k=7).jpg} 
        \caption{Мультипликативный шум} 
        \label{median_7:f} 
    \end{subfigure} 
    \caption{Результат применения медианной фильтрации к различным типам шумов при $k = 7$}
    \label{img:median_7} 
\end{figure}
\FloatBarrier

\subsubsection{Взвешенная медианная фильтрация}


\begin{figure}[ht!] 
    \centering
    \begin{subfigure}[b]{0.5\linewidth}
        \centering
        \includegraphics[width=0.95\linewidth]{../lewis-hine-taschen-main-3.jpg} 
        \caption{Исходное изображение} 
        \label{median2d_3:a} 
        \vspace{4ex}
    \end{subfigure}%%
    \begin{subfigure}[b]{0.5\linewidth}
      \centering
      \includegraphics[width=0.95\linewidth]{../Median_2D_Filter/Median_2DImpulse_noise_(k=3).jpg} 
      \caption{Импульсный шум} 
      \label{median2d_3:b} 
      \vspace{4ex}
    \end{subfigure}
    \begin{subfigure}[b]{0.5\linewidth}
      \centering
      \includegraphics[width=0.95\linewidth]{../Median_2D_Filter/Median_2DAdditive_noise_(k=3).jpg} 
      \caption{Аддитивный шум} 
      \label{median2d_3:c} 
      \vspace{4ex}
    \end{subfigure}%%
    \begin{subfigure}[b]{0.5\linewidth}
      \centering
      \includegraphics[width=0.95\linewidth]{../Median_2D_Filter/Median_2DGaussian_noise_(k=3).jpg} 
      \caption{Гауссов шум} 
      \label{median2d_3:d} 
      \vspace{4ex}
    \end{subfigure}
    \begin{subfigure}[b]{0.5\linewidth}
      \centering
      \includegraphics[width=0.95\linewidth]{../Median_2D_Filter/Median_2DPoisson_noise_(k=3).jpg} 
      \caption{Шум квантования} 
      \label{median2d_3:e}
    \end{subfigure}%% 
    \begin{subfigure}[b]{0.5\linewidth}
        \centering
        \includegraphics[width=0.95\linewidth]{../Median_2D_Filter/Median_2DSpeckle_noise_(k=3).jpg} 
        \caption{Мультипликативный шум} 
        \label{median2d_3:f} 
    \end{subfigure} 
    \caption{Результат применения взвешенной медианной фильтрации к различным типам шумов при $k = 3$}
    \label{img:median2d_3} 
\end{figure}

\begin{figure}[ht!] 
    \centering
    \begin{subfigure}[b]{0.5\linewidth}
        \centering
        \includegraphics[width=0.95\linewidth]{../lewis-hine-taschen-main-3.jpg} 
        \caption{Исходное изображение} 
        \label{median2d_5:a} 
        \vspace{4ex}
    \end{subfigure}%%
    \begin{subfigure}[b]{0.5\linewidth}
      \centering
      \includegraphics[width=0.95\linewidth]{../Median_2D_Filter/Median_2DImpulse_noise_(k=5).jpg} 
      \caption{Импульсный шум} 
      \label{median2d_5:b} 
      \vspace{4ex}
    \end{subfigure}
    \begin{subfigure}[b]{0.5\linewidth}
      \centering
      \includegraphics[width=0.95\linewidth]{../Median_2D_Filter/Median_2DAdditive_noise_(k=5).jpg} 
      \caption{Аддитивный шум} 
      \label{median2d_5:c} 
      \vspace{4ex}
    \end{subfigure}%%
    \begin{subfigure}[b]{0.5\linewidth}
      \centering
      \includegraphics[width=0.95\linewidth]{../Median_2D_Filter/Median_2DGaussian_noise_(k=5).jpg} 
      \caption{Гауссов шум} 
      \label{median2d_5:d} 
      \vspace{4ex}
    \end{subfigure}
    \begin{subfigure}[b]{0.5\linewidth}
      \centering
      \includegraphics[width=0.95\linewidth]{../Median_2D_Filter/Median_2DPoisson_noise_(k=5).jpg} 
      \caption{Шум квантования} 
      \label{median2d_5:e}
    \end{subfigure}%% 
    \begin{subfigure}[b]{0.5\linewidth}
        \centering
        \includegraphics[width=0.95\linewidth]{../Median_2D_Filter/Median_2DSpeckle_noise_(k=5).jpg} 
        \caption{Мультипликативный шум} 
        \label{median2d_5:f} 
    \end{subfigure} 
    \caption{Результат применения взвешенной медианной фильтрации к различным типам шумов при $k = 5$}
    \label{img:median2d_5} 
\end{figure}

\begin{figure}[ht!] 
    \centering
    \begin{subfigure}[b]{0.5\linewidth}
        \centering
        \includegraphics[width=0.95\linewidth]{../lewis-hine-taschen-main-3.jpg} 
        \caption{Исходное изображение} 
        \label{median2d_7:a} 
        \vspace{4ex}
    \end{subfigure}%%
    \begin{subfigure}[b]{0.5\linewidth}
      \centering
      \includegraphics[width=0.95\linewidth]{../Median_2D_Filter/Median_2DImpulse_noise_(k=7).jpg} 
      \caption{Импульсный шум} 
      \label{median2d_7:b} 
      \vspace{4ex}
    \end{subfigure}
    \begin{subfigure}[b]{0.5\linewidth}
      \centering
      \includegraphics[width=0.95\linewidth]{../Median_2D_Filter/Median_2DAdditive_noise_(k=7).jpg} 
      \caption{Аддитивный шум} 
      \label{median2d_7:c} 
      \vspace{4ex}
    \end{subfigure}%%
    \begin{subfigure}[b]{0.5\linewidth}
      \centering
      \includegraphics[width=0.95\linewidth]{../Median_2D_Filter/Median_2DGaussian_noise_(k=7).jpg} 
      \caption{Гауссов шум} 
      \label{median2d_7:d} 
      \vspace{4ex}
    \end{subfigure}
    \begin{subfigure}[b]{0.5\linewidth}
      \centering
      \includegraphics[width=0.95\linewidth]{../Median_2D_Filter/Median_2DPoisson_noise_(k=7).jpg} 
      \caption{Шум квантования} 
      \label{median2d_7:e}
    \end{subfigure}%% 
    \begin{subfigure}[b]{0.5\linewidth}
        \centering
        \includegraphics[width=0.95\linewidth]{../Median_2D_Filter/Median_2DSpeckle_noise_(k=7).jpg} 
        \caption{Мультипликативный шум} 
        \label{median2d_7:f} 
    \end{subfigure} 
    \caption{Результат применения взвешенной медианной фильтрации к различным типам шумов при $k = 7$}
    \label{img:median2d_7} 
\end{figure}
\FloatBarrier


\subsubsection{Адаптивная медианная фильтрация}

Обозначим через $z_{min},~z_{max},~z_{med}$ минимальное, максимальное и медианное значения интенсивностей в окне, $z_{i,j}$ — значение интенсивности пикселя с координатами $(i, j), s_{max}$ — максимально допустимый размер окна.
Алгоритм адаптивной медианной фильтрации
состоит из следующих шагов: \\
\begin{enumerate}
    \item Вычисление значений $z_{min}, z_{max}, z_{med}, A_1 =  z_{med} - z_{min}, A_2 = z_{med} - z_{max}$ пикселя $(i, j)$ в заданном окне. \\
    (a) Если $A_1 > 0 $ и $A_2 < 0$, перейти на шаг 2. В противном случае увеличить размер окна. \\
    (b) Если текущий размер окна $s \le s_{max}$, повторить шаг 1. В
противном случае результат фильтрации равен $z_{i,j}$. 
    \item Вычисление значений $B_1 = z_{i,j} - z_{min}, B_2 = z_{i,j} - z_{max}$. \\
    (a) Если $B_1 > 0$ и $B_2 < 0$, результат фильтрации равен $z_{i,j}$.
    В противном случае результат фильтрации равен $z_{med}$.

    \item Изменение координат $(i,j)$.
    (a) Если не достигнут предел изображения, перейти на шаг 1. В противном случае фильтрация окончена.
\end{enumerate}


Основной идеей является увеличение размера окна до тех пор,
пока алгоритм не найдет медианное значение, не являющееся импульсным шумом, или пока не достигнет максимального размера
окна. В последнем случае алгоритм вернет величину $z_{i,j}$.

\begin{figure}[ht!] 
  \centering
  \begin{subfigure}[b]{0.5\linewidth}
      \centering
      \includegraphics[width=0.95\linewidth]{../lewis-hine-taschen-main-3.jpg} 
      \caption{Исходное изображение} 
      \vspace{4ex}
  \end{subfigure}%%
  \begin{subfigure}[b]{0.5\linewidth}
    \centering
    \includegraphics[width=0.95\linewidth]{../Adaptive_Median_Filter/Adaptive_Median_Impulse_noise_k=3.jpg} 
    \caption{Импульсный шум} 
    \vspace{4ex}
  \end{subfigure}
  \begin{subfigure}[b]{0.5\linewidth}
    \centering
    \includegraphics[width=0.95\linewidth]{../Adaptive_Median_Filter/Adaptive_Median_Additive_noise_k=3.jpg} 
    \caption{Аддитивный шум} 
    \vspace{4ex}
  \end{subfigure}%%
  \begin{subfigure}[b]{0.5\linewidth}
    \centering
    \includegraphics[width=0.95\linewidth]{../Adaptive_Median_Filter/Adaptive_Median_Gaussian_noise_k=3.jpg} 
    \caption{Гауссов шум} 
    \vspace{4ex}
  \end{subfigure}
  \begin{subfigure}[b]{0.5\linewidth}
    \centering
    \includegraphics[width=0.95\linewidth]{../Adaptive_Median_Filter/Adaptive_Median_Poisson_noise_k=3.jpg} 
    \caption{Шум квантования} 
  \end{subfigure}%% 
  \begin{subfigure}[b]{0.5\linewidth}
      \centering
      \includegraphics[width=0.95\linewidth]{../Adaptive_Median_Filter/Adaptive_Median_Speckle_noise_k=3.jpg} 
      \caption{Мультипликативный шум} 
  \end{subfigure} 
  \caption{Результат применения адаптивной медианной фильтрации к различным типам шумов при $k = 3$}
\end{figure}

\begin{figure}[ht!] 
  \centering
  \begin{subfigure}[b]{0.5\linewidth}
      \centering
      \includegraphics[width=0.95\linewidth]{../lewis-hine-taschen-main-3.jpg} 
      \caption{Исходное изображение} 
      \vspace{4ex}
  \end{subfigure}%%
  \begin{subfigure}[b]{0.5\linewidth}
    \centering
    \includegraphics[width=0.95\linewidth]{../Adaptive_Median_Filter/Adaptive_Median_Impulse_noise_k=5.jpg} 
    \caption{Импульсный шум} 
    \vspace{4ex}
  \end{subfigure}
  \begin{subfigure}[b]{0.5\linewidth}
    \centering
    \includegraphics[width=0.95\linewidth]{../Adaptive_Median_Filter/Adaptive_Median_Additive_noise_k=5.jpg} 
    \caption{Аддитивный шум} 
    \vspace{4ex}
  \end{subfigure}%%
  \begin{subfigure}[b]{0.5\linewidth}
    \centering
    \includegraphics[width=0.95\linewidth]{../Adaptive_Median_Filter/Adaptive_Median_Gaussian_noise_k=5.jpg} 
    \caption{Гауссов шум} 
    \vspace{4ex}
  \end{subfigure}
  \begin{subfigure}[b]{0.5\linewidth}
    \centering
    \includegraphics[width=0.95\linewidth]{../Adaptive_Median_Filter/Adaptive_Median_Poisson_noise_k=5.jpg} 
    \caption{Шум квантования} 
  \end{subfigure}%% 
  \begin{subfigure}[b]{0.5\linewidth}
      \centering
      \includegraphics[width=0.95\linewidth]{../Adaptive_Median_Filter/Adaptive_Median_Speckle_noise_k=5.jpg} 
      \caption{Мультипликативный шум} 
  \end{subfigure} 
  \caption{Результат применения адаптивной медианной фильтрации к различным типам шумов при $k = 5$}
\end{figure}

\begin{figure}[ht!] 
  \centering
  \begin{subfigure}[b]{0.5\linewidth}
      \centering
      \includegraphics[width=0.95\linewidth]{../lewis-hine-taschen-main-3.jpg} 
      \caption{Исходное изображение} 
      \vspace{4ex}
  \end{subfigure}%%
  \begin{subfigure}[b]{0.5\linewidth}
    \centering
    \includegraphics[width=0.95\linewidth]{../Adaptive_Median_Filter/Adaptive_Median_Impulse_noise_k=7.jpg} 
    \caption{Импульсный шум} 
    \vspace{4ex}
  \end{subfigure}
  \begin{subfigure}[b]{0.5\linewidth}
    \centering
    \includegraphics[width=0.95\linewidth]{../Adaptive_Median_Filter/Adaptive_Median_Additive_noise_k=7.jpg} 
    \caption{Аддитивный шум} 
    \vspace{4ex}
  \end{subfigure}%%
  \begin{subfigure}[b]{0.5\linewidth}
    \centering
    \includegraphics[width=0.95\linewidth]{../Adaptive_Median_Filter/Adaptive_Median_Gaussian_noise_k=7.jpg} 
    \caption{Гауссов шум} 
    \vspace{4ex}
  \end{subfigure}
  \begin{subfigure}[b]{0.5\linewidth}
    \centering
    \includegraphics[width=0.95\linewidth]{../Adaptive_Median_Filter/Adaptive_Median_Poisson_noise_k=7.jpg} 
    \caption{Шум квантования} 
  \end{subfigure}%% 
  \begin{subfigure}[b]{0.5\linewidth}
      \centering
      \includegraphics[width=0.95\linewidth]{../Adaptive_Median_Filter/Adaptive_Median_Speckle_noise_k=7.jpg} 
      \caption{Мультипликативный шум} 
  \end{subfigure} 
  \caption{Результат применения адаптивной медианной фильтрации к различным типам шумов при $k = 3$}
\end{figure}


\begin{figure}[ht!] 
  \centering
  \begin{subfigure}[b]{0.5\linewidth}
      \centering
      \includegraphics[width=0.95\linewidth]{../lewis-hine-taschen-main-3.jpg} 
      \caption{Исходное изображение} 
      \vspace{4ex}
  \end{subfigure}%%
  \begin{subfigure}[b]{0.5\linewidth}
    \centering
    \includegraphics[width=0.95\linewidth]{../Adaptive_Median_Filter/Adaptive_Median_Impulse_noise_k=9.jpg} 
    \caption{Импульсный шум} 
    \vspace{4ex}
  \end{subfigure}
  \begin{subfigure}[b]{0.5\linewidth}
    \centering
    \includegraphics[width=0.95\linewidth]{../Adaptive_Median_Filter/Adaptive_Median_Additive_noise_k=9.jpg} 
    \caption{Аддитивный шум} 
    \vspace{4ex}
  \end{subfigure}%%
  \begin{subfigure}[b]{0.5\linewidth}
    \centering
    \includegraphics[width=0.95\linewidth]{../Adaptive_Median_Filter/Adaptive_Median_Gaussian_noise_k=9.jpg} 
    \caption{Гауссов шум} 
    \vspace{4ex}
  \end{subfigure}
  \begin{subfigure}[b]{0.5\linewidth}
    \centering
    \includegraphics[width=0.95\linewidth]{../Adaptive_Median_Filter/Adaptive_Median_Poisson_noise_k=9.jpg} 
    \caption{Шум квантования} 
  \end{subfigure}%% 
  \begin{subfigure}[b]{0.5\linewidth}
      \centering
      \includegraphics[width=0.95\linewidth]{../Adaptive_Median_Filter/Adaptive_Median_Speckle_noise_k=9.jpg} 
      \caption{Мультипликативный шум} 
  \end{subfigure} 
  \caption{Результат применения адаптивной медианной фильтрации к различным типам шумов при $k = 9$}
\end{figure}
\FloatBarrier

\subsubsection{Ранговая фильтрация}

\begin{figure}[ht!] 
    \centering
    \begin{subfigure}[b]{0.5\linewidth}
        \centering
        \includegraphics[width=0.95\linewidth]{../lewis-hine-taschen-main-3.jpg} 
        \caption{Исходное изображение} 
        \label{rang_3_1:a} 
        \vspace{4ex}
    \end{subfigure}%%
    \begin{subfigure}[b]{0.5\linewidth}
      \centering
      \includegraphics[width=0.95\linewidth]{../Rang_Filter/Rang_Impulse_noise_(k=3,r=1).jpg} 
      \caption{Импульсный шум} 
      \label{rang_3_1:b} 
      \vspace{4ex}
    \end{subfigure}
    \begin{subfigure}[b]{0.5\linewidth}
      \centering
      \includegraphics[width=0.95\linewidth]{../Rang_Filter/Rang_Additive_noise_(k=3,r=1).jpg} 
      \caption{Аддитивный шум} 
      \label{rang_3_1:c} 
      \vspace{4ex}
    \end{subfigure}%%
    \begin{subfigure}[b]{0.5\linewidth}
      \centering
      \includegraphics[width=0.95\linewidth]{../Rang_Filter/Rang_Gaussian_noise_(k=3,r=1).jpg} 
      \caption{Гауссов шум} 
      \label{rang_3_1:d} 
      \vspace{4ex}
    \end{subfigure}
    \begin{subfigure}[b]{0.5\linewidth}
      \centering
      \includegraphics[width=0.95\linewidth]{../Rang_Filter/Rang_Poisson_noise_(k=3,r=1).jpg} 
      \caption{Шум квантования} 
      \label{rang_3_1:e}
    \end{subfigure}%% 
    \begin{subfigure}[b]{0.5\linewidth}
        \centering
        \includegraphics[width=0.95\linewidth]{../Rang_Filter/Rang_Speckle_noise_(k=3,r=1).jpg} 
        \caption{Мультипликативный шум} 
        \label{rang_3_1:f} 
    \end{subfigure} 
    \caption{Результат применения ранговой фильтрации к различным типам шумов при $k = 3, r = 1$}
    \label{img:rang_3_1} 
\end{figure}

\begin{figure}[ht!] 
    \centering
    \begin{subfigure}[b]{0.5\linewidth}
        \centering
        \includegraphics[width=0.95\linewidth]{../lewis-hine-taschen-main-3.jpg} 
        \caption{Исходное изображение} 
        \label{rang_3_5:a} 
        \vspace{4ex}
    \end{subfigure}%%
    \begin{subfigure}[b]{0.5\linewidth}
      \centering
      \includegraphics[width=0.95\linewidth]{../Rang_Filter/Rang_Impulse_noise_(k=3,r=5).jpg} 
      \caption{Импульсный шум} 
      \label{rang_3_5:b} 
      \vspace{4ex}
    \end{subfigure}
    \begin{subfigure}[b]{0.5\linewidth}
      \centering
      \includegraphics[width=0.95\linewidth]{../Rang_Filter/Rang_Additive_noise_(k=3,r=5).jpg} 
      \caption{Аддитивный шум} 
      \label{rang_3_5:c} 
      \vspace{4ex}
    \end{subfigure}%%
    \begin{subfigure}[b]{0.5\linewidth}
      \centering
      \includegraphics[width=0.95\linewidth]{../Rang_Filter/Rang_Gaussian_noise_(k=3,r=5).jpg} 
      \caption{Гауссов шум} 
      \label{rang_3_5:d} 
      \vspace{4ex}
    \end{subfigure}
    \begin{subfigure}[b]{0.5\linewidth}
      \centering
      \includegraphics[width=0.95\linewidth]{../Rang_Filter/Rang_Poisson_noise_(k=3,r=5).jpg} 
      \caption{Шум квантования} 
      \label{rang_3_5:e}
    \end{subfigure}%% 
    \begin{subfigure}[b]{0.5\linewidth}
        \centering
        \includegraphics[width=0.95\linewidth]{../Rang_Filter/Rang_Speckle_noise_(k=3,r=5).jpg} 
        \caption{Мультипликативный шум} 
        \label{rang_3_5:f} 
    \end{subfigure} 
    \caption{Результат применения ранговой фильтрации к различным типам шумов при $k = 3, r = 5$}
    \label{img:rang_3_5} 
\end{figure}

\begin{figure}[ht!] 
    \centering
    \begin{subfigure}[b]{0.5\linewidth}
        \centering
        \includegraphics[width=0.95\linewidth]{../lewis-hine-taschen-main-3.jpg} 
        \caption{Исходное изображение} 
        \label{rang_3_9:a} 
        \vspace{4ex}
    \end{subfigure}%%
    \begin{subfigure}[b]{0.5\linewidth}
      \centering
      \includegraphics[width=0.95\linewidth]{../Rang_Filter/Rang_Impulse_noise_(k=3,r=9).jpg} 
      \caption{Импульсный шум} 
      \label{rang_3_9:b} 
      \vspace{4ex}
    \end{subfigure}
    \begin{subfigure}[b]{0.5\linewidth}
      \centering
      \includegraphics[width=0.95\linewidth]{../Rang_Filter/Rang_Additive_noise_(k=3,r=9).jpg} 
      \caption{Аддитивный шум} 
      \label{rang_3_9:c} 
      \vspace{4ex}
    \end{subfigure}%%
    \begin{subfigure}[b]{0.5\linewidth}
      \centering
      \includegraphics[width=0.95\linewidth]{../Rang_Filter/Rang_Gaussian_noise_(k=3,r=9).jpg} 
      \caption{Гауссов шум} 
      \label{rang_3_9:d} 
      \vspace{4ex}
    \end{subfigure}
    \begin{subfigure}[b]{0.5\linewidth}
      \centering
      \includegraphics[width=0.95\linewidth]{../Rang_Filter/Rang_Poisson_noise_(k=3,r=9).jpg} 
      \caption{Шум квантования} 
      \label{rang_3_9:e}
    \end{subfigure}%% 
    \begin{subfigure}[b]{0.5\linewidth}
        \centering
        \includegraphics[width=0.95\linewidth]{../Rang_Filter/Rang_Speckle_noise_(k=3,r=9).jpg} 
        \caption{Мультипликативный шум} 
        \label{rang_3_9:f} 
    \end{subfigure} 
    \caption{Результат применения ранговой фильтрации к различным типам шумов при $k = 3, r = 9$}
    \label{img:rang_3_9} 
\end{figure}




\begin{figure}[ht!] 
    \centering
    \begin{subfigure}[b]{0.5\linewidth}
        \centering
        \includegraphics[width=0.95\linewidth]{../lewis-hine-taschen-main-3.jpg} 
        \caption{Исходное изображение} 
        \label{rang_5_1:a} 
        \vspace{4ex}
    \end{subfigure}%%
    \begin{subfigure}[b]{0.5\linewidth}
      \centering
      \includegraphics[width=0.95\linewidth]{../Rang_Filter/Rang_Impulse_noise_(k=5,r=1).jpg} 
      \caption{Импульсный шум} 
      \label{rang_5_1:b} 
      \vspace{4ex}
    \end{subfigure}
    \begin{subfigure}[b]{0.5\linewidth}
      \centering
      \includegraphics[width=0.95\linewidth]{../Rang_Filter/Rang_Additive_noise_(k=5,r=1).jpg} 
      \caption{Аддитивный шум} 
      \label{rang_5_1:c} 
      \vspace{4ex}
    \end{subfigure}%%
    \begin{subfigure}[b]{0.5\linewidth}
      \centering
      \includegraphics[width=0.95\linewidth]{../Rang_Filter/Rang_Gaussian_noise_(k=5,r=1).jpg} 
      \caption{Гауссов шум} 
      \label{rang_5_1:d} 
      \vspace{4ex}
    \end{subfigure}
    \begin{subfigure}[b]{0.5\linewidth}
      \centering
      \includegraphics[width=0.95\linewidth]{../Rang_Filter/Rang_Poisson_noise_(k=5,r=1).jpg} 
      \caption{Шум квантования} 
      \label{rang_5_1:e}
    \end{subfigure}%% 
    \begin{subfigure}[b]{0.5\linewidth}
        \centering
        \includegraphics[width=0.95\linewidth]{../Rang_Filter/Rang_Speckle_noise_(k=5,r=1).jpg} 
        \caption{Мультипликативный шум} 
        \label{rang_5_1:f} 
    \end{subfigure} 
    \caption{Результат применения ранговой фильтрации к различным типам шумов при $k = 5, r = 1$}
    \label{img:rang_5_1} 
\end{figure}

\begin{figure}[ht!] 
    \centering
    \begin{subfigure}[b]{0.5\linewidth}
        \centering
        \includegraphics[width=0.95\linewidth]{../lewis-hine-taschen-main-3.jpg} 
        \caption{Исходное изображение} 
        \label{rang_5_14:a} 
        \vspace{4ex}
    \end{subfigure}%%
    \begin{subfigure}[b]{0.5\linewidth}
      \centering
      \includegraphics[width=0.95\linewidth]{../Rang_Filter/Rang_Impulse_noise_(k=5,r=14).jpg} 
      \caption{Импульсный шум} 
      \label{rang_5_14:b} 
      \vspace{4ex}
    \end{subfigure}
    \begin{subfigure}[b]{0.5\linewidth}
      \centering
      \includegraphics[width=0.95\linewidth]{../Rang_Filter/Rang_Additive_noise_(k=5,r=14).jpg} 
      \caption{Аддитивный шум} 
      \label{rang_5_14:c} 
      \vspace{4ex}
    \end{subfigure}%%
    \begin{subfigure}[b]{0.5\linewidth}
      \centering
      \includegraphics[width=0.95\linewidth]{../Rang_Filter/Rang_Gaussian_noise_(k=5,r=14).jpg} 
      \caption{Гауссов шум} 
      \label{rang_5_14:d} 
      \vspace{4ex}
    \end{subfigure}
    \begin{subfigure}[b]{0.5\linewidth}
      \centering
      \includegraphics[width=0.95\linewidth]{../Rang_Filter/Rang_Poisson_noise_(k=5,r=14).jpg} 
      \caption{Шум квантования} 
      \label{rang_5_14:e}
    \end{subfigure}%% 
    \begin{subfigure}[b]{0.5\linewidth}
        \centering
        \includegraphics[width=0.95\linewidth]{../Rang_Filter/Rang_Speckle_noise_(k=5,r=14).jpg} 
        \caption{Мультипликативный шум} 
        \label{rang_5_14:f} 
    \end{subfigure} 
    \caption{Результат применения ранговой фильтрации к различным типам шумов при $k = 5, r = 14$}
    \label{img:rang_5_14} 
\end{figure}

\begin{figure}[ht!] 
    \centering
    \begin{subfigure}[b]{0.5\linewidth}
        \centering
        \includegraphics[width=0.95\linewidth]{../lewis-hine-taschen-main-3.jpg} 
        \caption{Исходное изображение} 
        \label{rang_5_25:a} 
        \vspace{4ex}
    \end{subfigure}%%
    \begin{subfigure}[b]{0.5\linewidth}
      \centering
      \includegraphics[width=0.95\linewidth]{../Rang_Filter/Rang_Impulse_noise_(k=5,r=25).jpg} 
      \caption{Импульсный шум} 
      \label{rang_5_25:b} 
      \vspace{4ex}
    \end{subfigure}
    \begin{subfigure}[b]{0.5\linewidth}
      \centering
      \includegraphics[width=0.95\linewidth]{../Rang_Filter/Rang_Additive_noise_(k=5,r=25).jpg} 
      \caption{Аддитивный шум} 
      \label{rang_5_25:c} 
      \vspace{4ex}
    \end{subfigure}%%
    \begin{subfigure}[b]{0.5\linewidth}
      \centering
      \includegraphics[width=0.95\linewidth]{../Rang_Filter/Rang_Gaussian_noise_(k=5,r=25).jpg} 
      \caption{Гауссов шум} 
      \label{rang_5_25:d} 
      \vspace{4ex}
    \end{subfigure}
    \begin{subfigure}[b]{0.5\linewidth}
      \centering
      \includegraphics[width=0.95\linewidth]{../Rang_Filter/Rang_Poisson_noise_(k=5,r=25).jpg} 
      \caption{Шум квантования} 
      \label{rang_5_25:e}
    \end{subfigure}%% 
    \begin{subfigure}[b]{0.5\linewidth}
        \centering
        \includegraphics[width=0.95\linewidth]{../Rang_Filter/Rang_Speckle_noise_(k=5,r=25).jpg} 
        \caption{Мультипликативный шум} 
        \label{rang_5_25:f} 
    \end{subfigure} 
    \caption{Результат применения ранговой фильтрации к различным типам шумов при $k = 5, r = 25$}
    \label{img:rang_5_25} 
\end{figure}
\FloatBarrier

\subsubsection{Винеровская фильтрация}

\begin{figure}[ht!] 
    \centering
    \begin{subfigure}[b]{0.5\linewidth}
        \centering
        \includegraphics[width=0.95\linewidth]{../lewis-hine-taschen-main-3.jpg} 
        \caption{Исходное изображение} 
        \label{wiener_3:a} 
        \vspace{4ex}
    \end{subfigure}%%
    \begin{subfigure}[b]{0.5\linewidth}
      \centering
      \includegraphics[width=0.95\linewidth]{../Wiener_Filter/Wiener_Impulse_noise_(k=3).jpg} 
      \caption{Импульсный шум} 
      \label{weiner_3:b} 
      \vspace{4ex}
    \end{subfigure}
    \begin{subfigure}[b]{0.5\linewidth}
      \centering
      \includegraphics[width=0.95\linewidth]{../Wiener_Filter/Wiener_Additive_noise_(k=3).jpg} 
      \caption{Аддитивный шум} 
      \label{weiner_3:c} 
      \vspace{4ex}
    \end{subfigure}%%
    \begin{subfigure}[b]{0.5\linewidth}
      \centering
      \includegraphics[width=0.95\linewidth]{../Wiener_Filter/Wiener_Gaussian_noise_(k=3).jpg} 
      \caption{Гауссов шум} 
      \label{weiner_3:d} 
      \vspace{4ex}
    \end{subfigure}
    \begin{subfigure}[b]{0.5\linewidth}
      \centering
      \includegraphics[width=0.95\linewidth]{../Wiener_Filter/Wiener_Poisson_noise_(k=3).jpg} 
      \caption{Шум квантования} 
      \label{weiner_3:e}
    \end{subfigure}%% 
    \begin{subfigure}[b]{0.5\linewidth}
        \centering
        \includegraphics[width=0.95\linewidth]{../Wiener_Filter/Wiener_Speckle_noise_(k=3).jpg} 
        \caption{Мультипликативный шум} 
        \label{weiner_3:f} 
    \end{subfigure} 
    \caption{Результат применения Винеровской фильтрации к различным типам шумов при $k = 3$}
    \label{img:weiner_3} 
\end{figure}

\begin{figure}[ht!] 
    \centering
    \begin{subfigure}[b]{0.5\linewidth}
        \centering
        \includegraphics[width=0.95\linewidth]{../lewis-hine-taschen-main-3.jpg} 
        \caption{Исходное изображение} 
        \label{wiener_5:a} 
        \vspace{4ex}
    \end{subfigure}%%
    \begin{subfigure}[b]{0.5\linewidth}
      \centering
      \includegraphics[width=0.95\linewidth]{../Wiener_Filter/Wiener_Impulse_noise_(k=5).jpg} 
      \caption{Импульсный шум} 
      \label{weiner_5:b} 
      \vspace{4ex}
    \end{subfigure}
    \begin{subfigure}[b]{0.5\linewidth}
      \centering
      \includegraphics[width=0.95\linewidth]{../Wiener_Filter/Wiener_Additive_noise_(k=5).jpg} 
      \caption{Аддитивный шум} 
      \label{weiner_5:c} 
      \vspace{4ex}
    \end{subfigure}%%
    \begin{subfigure}[b]{0.5\linewidth}
      \centering
      \includegraphics[width=0.95\linewidth]{../Wiener_Filter/Wiener_Gaussian_noise_(k=5).jpg} 
      \caption{Гауссов шум} 
      \label{weiner_5:d} 
      \vspace{4ex}
    \end{subfigure}
    \begin{subfigure}[b]{0.5\linewidth}
      \centering
      \includegraphics[width=0.95\linewidth]{../Wiener_Filter/Wiener_Poisson_noise_(k=5).jpg} 
      \caption{Шум квантования} 
      \label{weiner_5:e}
    \end{subfigure}%% 
    \begin{subfigure}[b]{0.5\linewidth}
        \centering
        \includegraphics[width=0.95\linewidth]{../Wiener_Filter/Wiener_Speckle_noise_(k=5).jpg} 
        \caption{Мультипликативный шум} 
        \label{weiner_5:f} 
    \end{subfigure} 
    \caption{Результат применения Винеровской фильтрации к различным типам шумов при $k = 5$}
    \label{img:weiner_5} 
\end{figure}

\begin{figure}[ht!] 
    \centering
    \begin{subfigure}[b]{0.5\linewidth}
        \centering
        \includegraphics[width=0.95\linewidth]{../lewis-hine-taschen-main-3.jpg} 
        \caption{Исходное изображение} 
        \label{wiener_7:a} 
        \vspace{4ex}
    \end{subfigure}%%
    \begin{subfigure}[b]{0.5\linewidth}
      \centering
      \includegraphics[width=0.95\linewidth]{../Wiener_Filter/Wiener_Impulse_noise_(k=7).jpg} 
      \caption{Импульсный шум} 
      \label{weiner_7:b} 
      \vspace{4ex}
    \end{subfigure}
    \begin{subfigure}[b]{0.5\linewidth}
      \centering
      \includegraphics[width=0.95\linewidth]{../Wiener_Filter/Wiener_Additive_noise_(k=7).jpg} 
      \caption{Аддитивный шум} 
      \label{weiner_7:c} 
      \vspace{4ex}
    \end{subfigure}%%
    \begin{subfigure}[b]{0.5\linewidth}
      \centering
      \includegraphics[width=0.95\linewidth]{../Wiener_Filter/Wiener_Gaussian_noise_(k=7).jpg} 
      \caption{Гауссов шум} 
      \label{weiner_7:d} 
      \vspace{4ex}
    \end{subfigure}
    \begin{subfigure}[b]{0.5\linewidth}
      \centering
      \includegraphics[width=0.95\linewidth]{../Wiener_Filter/Wiener_Poisson_noise_(k=7).jpg} 
      \caption{Шум квантования} 
      \label{weiner_7:e}
    \end{subfigure}%% 
    \begin{subfigure}[b]{0.5\linewidth}
        \centering
        \includegraphics[width=0.95\linewidth]{../Wiener_Filter/Wiener_Speckle_noise_(k=7).jpg} 
        \caption{Мультипликативный шум} 
        \label{weiner_7:f} 
    \end{subfigure} 
    \caption{Результат применения Винеровской фильтрации к различным типам шумов при $k = 7$}
    \label{img:weiner_7} 
\end{figure}
\FloatBarrier


\subsection{Высокочастотная фильтрация}
\subsubsection{Фильтр Робертса}

\begin{figure}[ht!]
  \centering
  \includegraphics[width=\textwidth]{../Edge_Detection/Robertson_Smoking_boys.jpg}
  \caption{Фильтр Робертса}
  \caption{Результат применения фильтра Робертса}
  \label{img:roberts}  
\end{figure}
\FloatBarrier

Наблюдаем, что изображение стало практически черным, но, при этом, границы объектов на изображении различимы. 

\subsubsection{Фильтр Превитта}

\begin{figure}[ht!]
  \centering
  \includegraphics[width=\textwidth]{../Edge_Detection/Prewitt_Smoking_boys.jpg}
  \caption{Фильтр Превитта}
  \caption{Результат применения фильтра Превитта}
  \label{img:prewitt}  
\end{figure}
\FloatBarrier

В отличие от прошлого фильтра, границы объектов заметны отчетливо, но, в то же время, присутствуют артефакты -- белые пиксели там, где нет явных границ. 
\subsubsection{Фильтр Собела}

\begin{figure}[ht!]
  \centering
  \includegraphics[width=\textwidth]{../Edge_Detection/Sobel_Smoking_boys.jpg}
  \caption{Фильтр Собела}
  \caption{Результат применения фильтра Собела}
  \label{img:sobel}  
\end{figure}
\FloatBarrier

Границы стали еще более выраженными, количество артефактов увеличилось. 

\subsubsection{Фильтр Лапласа}

\begin{figure}[ht!]
  \centering
  \includegraphics[width=\textwidth]{../Edge_Detection/Laplassian_Smoking_boys.jpg}
  \caption{Фильтр Лапласа}
  \caption{Результат применения фильтра Лапласа}
  \label{img:laplassian}  
\end{figure}
\FloatBarrier

Границы выделились наиболее четко, количество артефактов уменьшилось.

\subsubsection{Алгоритм Кэнни}
\begin{figure}[ht!]
  \centering
  \includegraphics[width=\textwidth]{../Edge_Detection/Canny_Smoking_boys.jpg}
  \caption{Фильтр Лапласа}
  \caption{Результат применения алгоритма Кэнни}
  \label{img:canny}  
\end{figure}
\FloatBarrier

Наблюдаем наиболее яркие границы объектов. 

\section{Выводы}


В ходе выполнения лабораторной работы были изучены основные способы фильтрации изображений от шумов и выделения контуров. Были применены различные методы фильтрации, которые мы анализировали в процессе работы.

Таким образом, фильтрация изображений и выделение контуров играют важную роль в обработке изображений, позволяя улучшить их качество и провести анализ содержимого более эффективно.

\section{Ответы на вопросы}

\newcounter{question}
\setcounter{question}{0}

\newcommand{\question}[1]{\item[Q\refstepcounter{question}\thequestion.] #1}
\newcommand{\answer}[1]{\item[A\thequestion.] #1}

\begin{itemize}

\question{В чем заключаются основные недостатки адаптивных методов фильтрации изображений?}
\answer{Адаптивные фильтры обычно более сложны, чем неадаптивные. Они требуют больше вычислительных ресурсов и времени для корректировки своих параметров в процессе фильтрации. 

Несмотря на то, что адаптивные фильтры хорошо уменьшают шум, они иногда могут размывать границы на изображении. 

Эффективность адаптивных фильтров во многом зависит от выбора параметров. Выбор подходящих параметров для данного изображения или набора изображений может быть сложной задачей.

}

\question{При каких значениях параметра $Q$ контргармонический фильтр будет работать как арифметический, а при каких -- как гармонический?}
\answer{$Q$ — это порядок фильтра. Контргармонический фильтр является обобщением усредняющих фильтров. При $Q = 0$ фильтр превращается в арифметический, а при $Q = -1$ — в гармонический.}

\question{Какими операторами можно выделить границы на изображении?}
\answer{Можно выделить границы с использованием дифференциального оператора
Робертса.}

\question{Для чего на первом шаге выделения контуров, как правило, выполняется низкочастотная фильтрация?}
\answer{Для того, чтобы избавиться от возможных артефактов при выделении контуров. Низкочастотная фильтрация позволяет убрать мелкие детали и шумы на изображении, тем самым улучшить результат выделения контуров.}

\end{itemize}
